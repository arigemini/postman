\documentclass[senior,oneside]{UIUC}
\usepackage{indentfirst}
% Your name
  \Author{ARINDAM SAHA}

% For Senior and honors this is the year and month that you submit the thesis
% For Masters and PhD, this is your graduation date
  \Year{2012}
  \Month{May,}

% If you have a long title, split it between multiple lines using the \\ command
  \Title{CUSTOMER MIX ALGORITHM}

% Your research advisor
  \Advisor{DAN ROTH}

% For honors theses, enter the name of the honors Representative

% The text of your abstract
  \Abstract{A Customer Mix Algorithm (CMX) makes use of machine learning techniques to \
accurately determine whether messages intended to a specific customer are actually \ 
sent to a different customer. Detection of such messages has high value as this type of messages is considered \
harmful. In this project, we explore various machine learning algorithms and feature selection methods to determine the \
recipient of a message from the content, so that we can prevent messages from being sent to an incorrect recipient.}



% Acknowledge those who helped and supported you
  \Acknowledgments{
  	I am extremely thankful to Prof. Dan Roth for being my advisor and guiding me through this thesis. He has pointed me to various useful resources  that has increased my breadth of knowledge in the field of machine learning. Also, the dataset used to evaluate various training algorithms is the Enron Email Dataset and I am thankful to the CALO Project (A Cognitive Assistant that Learns and Organizes) for making that available. I am grateful to all the teachers at the University of Illinois at Urbana-Champaign for the various concepts in the field of Computer Science that I have learned over the past four years. Finally, I would like to thank my family for their encourgament and support.
  }


\begin{document}

 % Start page counting in roman numerals
 \frontmatter

 % This command makes the formal preliminary pages.
 % You can comment it out during the drafting process if you want to save paper.
 \makepreliminarypages

 % Make the table of contents.
 \tableofcontents

 % Start regular page counting at page 1
 \mainmatter

% OK. Everything is set up. Type your thesis here.

\chapter{Introduction}

\section{Customer Mix Algorithm}

A Customer Mix Algorithm (CMX) makes use of machine learning techniques to accurately determine whether messages intended to a specific customer are actually sent to a different customer. Such messages can be harmful for the sender and hence, being able to identify such messages are of high value. 

\section{Resources}

We use the Enron email dataset \cite{enrondataset} as our source of data. The next section presents various statistics about this large dataset of emails. \
We use SNoW (Sparse Network of Winnows) \cite{snow} to do training on the data. SNoW has well-known machine learning algorithms already implemented, \
namely Winnow, Perceptron and Naive Bayes, and is extremely efficient when training on large datasets. We also use the Illinois named entity tagger \cite{nertagger} to extract features from emails and use in  training. 

\chapter{Analysis of the dataset}

The input data for developing the Customer Mix Algorithm consists of the Enron data set \cite{enrondataset}. It is a massive collection of emails sent between the Enron employees and in this chapter we present various statistics about the data.

\section{Statistics}

The Enron dataset has  a total of 517424 emails across 150 accounts. Each account has messages grouped by various folder types (e.g. inbox, sent, etc...). We did an analysis of the distribution and the results are in the tables below.

\begin{table}[ht]
\caption{Emails per account statistics}
\centering
\begin{tabular}{c c c }
\hline \hline
Account & No. of emails & \% \\ [0.5ex]
%heading
\hline
kaminski-v & 28465 & 1 \\
dasovich-j & 28234 & 1 \\
kean-s & 25351 & 1 \\
mann-k & 23381 & 1 \\
jones-t & 19950 & 1 \\
shackleton-s & 18687 & 1 \\
taylor-m & 13875 & 1 \\
farmer-d & 13032 & 1 \\
germany-c & 12436 & 1 \\
beck-s & 11830 & 1 \\
symes-k & 10827 & 1 \\
nemec-g & 10655 & 1 \\
scott-s & 8022 & 1 \\
rogers-b & 8009 & 1 \\
bass-e & 7823 & 1 \\
sanders-r & 7329 & 1 \\
campbell-l & 6490 & 1 \\
shapiro-r & 6071 & 1 \\
guzman-m & 6054 & 1 \\
lay-k & 5937 & 1 \\
\hline
\end{tabular}
\end{table}

\begin{table}
\parbox{.45\linewidth}{
\centering
\begin{tabular}{c c c }
\hline \hline
Account & No. of emails & \% \\ [0.5ex]
%heading
\hline
kaminski-v & 28465 & 1 \\
dasovich-j & 28234 & 1 \\
kean-s & 25351 & 1 \\
mann-k & 23381 & 1 \\
jones-t & 19950 & 1 \\
shackleton-s & 18687 & 1 \\
taylor-m & 13875 & 1 \\
farmer-d & 13032 & 1 \\
germany-c & 12436 & 1 \\
beck-s & 11830 & 1 \\
symes-k & 10827 & 1 \\
nemec-g & 10655 & 1 \\
scott-s & 8022 & 1 \\
rogers-b & 8009 & 1 \\
bass-e & 7823 & 1 \\
sanders-r & 7329 & 1 \\
campbell-l & 6490 & 1 \\
shapiro-r & 6071 & 1 \\
guzman-m & 6054 & 1 \\
lay-k & 5937 & 1 \\
\hline
\end{tabular}
\caption{Emails per directory}
}
\hfill
\parbox{.45\linewidth}{
\centering
\begin{tabular}{c c c }
\hline \hline
Folder & No. of emails & \% \\ [0.5ex]
%heading
\hline
all\_documents & 128103 & 1 \\
discussion\_threads & 58609 & 1 \\
sent & 57653 & 1 \\
deleted\_items & 51356 & 1 \\
inbox & 44871 & 1 \\
sent\_items & 37935 & 1 \\
notes\_inbox & 36666 & 1 \\
\_sent\_mail & 30109 & 1 \\
calendar & 6133 & 1 \\
archiving & 4477 & 1 \\
\_americas & 4021 & 1 \\
personal & 2577 & 1 \\
attachments & 2026 & 1 \\
meetings & 1872 & 1 \\
c & 1656 & 1 \\
schedule\_crawler & 1398 & 1 \\
chris\_stokley & 1252 & 1 \\
logistics & 1192 & 1 \\
archive & 1179 & 1 \\
tw\_commercial\_group & 1159 & 1 \\
\hline
\end{tabular}
\caption{Emails per folder}
}
\end{table}

\chapter{Features}
\section{Email fields}
\section{Named Entities}
\section{TF-IDF}

\chapter{Learning Framework}
\section{Email filtering}
\section{Precomputation}
\section{Generating training data}
\section{Testing}

\chapter{Results}
\section{Methodology}
\section{Statistics}

\chapter{Conclusion}
\section{Summary}


% Make the bibliography.
% Enter your references in the BibTex file "references.bib"
% \bibliography{references}


\begin{thebibliography}{9}

\bibitem{ner}
  Lev Ratinov and Dan Roth, 2009,
  Design challenges and misconceptions in named entity recognition,
  \emph{Proceedings of the thirteenth Conference on Computational Natural Language Learning (CONLL)}.

\bibitem{idf}
Roberson, S.E. (2004),
Understanding inverse document frequency: On theoretical arguments for IDF,
 \emph{Journal of Documentation, Vol. 60, No. 5. (2004), pp. 503-520}

\bibitem{tfidf}
  Juan Ramos, 2003,
  \emph{Using TF-IDF to Determine World Relevance in Document Queries}

\bibitem{snow}
 Andy Carlson, Chad Cumby, Nick Rizzolo, Jeff Rosen, Mark Sammons and Dan Roth, 
 \emph{http://cogcomp.cs.illinois.edu/page/software\_view/SNoW}

\bibitem{nertagger}
Illinois Named Entity Tagger, 
\emph{http://cogcomp.cs.illinois.edu/page/software\_view/NETagger}

\bibitem{enrondataset}
  Enron dataset,
  \emph{http://www.cs.cmu.edu/\texttildelow{}enron/}
 
\end{thebibliography}

\end{document}
